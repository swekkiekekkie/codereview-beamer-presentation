\section{Defining Functions}

\subsection{Introduction to Defining Functions}
\begin{frame}[fragile]
\frametitle{Introduction to Defining Functions}
\begin{itemize}
    \item Functions are blocks of code that perform a specific task.
    \item Functions can take arguments and return values.
    \item Example in C\#:
\end{itemize}
\begin{lstlisting}
int Add(int a, int b) {
    return a + b;
}

int result = Add(3, 4);
Console.WriteLine(result); // Output: 7
\end{lstlisting}
\end{frame}

\subsection{Passing Arguments to Parameters}
\begin{frame}[fragile]
\frametitle{Passing Arguments to Parameters}
\begin{itemize}
    \item Arguments are values passed to a function when it is called.
    \item Parameters are variables that hold the values of the arguments.
    \item Example in C\#:
\end{itemize}
\begin{lstlisting}
void Greet(string name) {
    Console.WriteLine("Hello, " + name);
}

Greet("Alice"); // Output: Hello, Alice
\end{lstlisting}
\end{frame}

\subsection{Common Confusions}
\begin{frame}[fragile]
\frametitle{Common Confusions}
\begin{itemize}
    \item One common confusion is understanding how arguments are passed to parameters.
    \item When a function is called, the arguments are loaded into the variables declared by the parameters in the function signature.
    \item Example in C\#:
\end{itemize}
\begin{lstlisting}
void Swap(ref int a, ref int b) {
    int temp = a;
    a = b;
    b = temp;
}

int x = 5;
int y = 10;
Swap(ref x, ref y);
Console.WriteLine("x: " + x + ", y: " + y); // Output: x: 10, y: 5
\end{lstlisting}
\end{frame}

\subsection{Benefits of Defining Functions}
\begin{frame}
\frametitle{Benefits of Defining Functions}
\begin{itemize}
    \item Functions allow you to encapsulate code into reusable blocks.
    \item Functions make your code more modular and easier to maintain.
    \item Functions can help you avoid repeating the same code multiple times.
\end{itemize}
\end{frame}
