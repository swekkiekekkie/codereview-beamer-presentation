\section{Constants}

\subsection{Introduction to Constants}
\begin{frame}[fragile]
\frametitle{Introduction to Constants}
\begin{itemize}
    \item Constants are fixed values that do not change during the execution of a program.
    \item Constants can be used to make code more readable and maintainable.
    \item Example in C\#:
\end{itemize}
\begin{lstlisting}
const double Pi = 3.14159;
Console.WriteLine("Value of Pi: " + Pi); // Output: Value of Pi: 3.14159
\end{lstlisting}
\end{frame}

\subsection{Using Constants in Functions}
\begin{frame}[fragile]
\frametitle{Using Constants in Functions}
\begin{itemize}
    \item Constants can be passed as arguments to functions.
    \item Example in C\#:
\end{itemize}
\begin{lstlisting}
const double Pi = 3.14159;

void PrintCircleArea(double radius) {
    double area = Pi * radius * radius;
    Console.WriteLine("Area of the circle: " + area);
}

PrintCircleArea(5); // Output: Area of the circle: 78.53975
\end{lstlisting}
\end{frame}

\subsection{Benefits of Using Constants}
\begin{frame}
\frametitle{Benefits of Using Constants}
\begin{itemize}
    \item Improved code readability: Constants provide meaningful names for fixed values.
    \item Easier maintenance: Constants can be updated in one place, reducing the risk of errors.
    \item Prevention of accidental changes: Constants cannot be modified during program execution.
\end{itemize}
\end{frame}
