\section{Variables}

\subsection{Introduction to Variables}
\begin{frame}[fragile]
\frametitle{Introduction to Variables}
\begin{itemize}
    \item Variables are used to store data that can be changed during the execution of a program.
    \item Variables have a name, a type, and a value.
    \item Example in C\#:
\end{itemize}
\begin{lstlisting}
int age = 25;
Console.WriteLine("Age: " + age); // Output: Age: 25
\end{lstlisting}
\end{frame}

\subsection{Using Variables in Functions}
\begin{frame}[fragile]
\frametitle{Using Variables in Functions}
\begin{itemize}
    \item Variables can be passed as arguments to functions.
    \item Example in C\#:
\end{itemize}
\begin{lstlisting}
int age = 25;

void PrintAge(int age) {
    Console.WriteLine("Age: " + age);
}

PrintAge(age); // Output: Age: 25
\end{lstlisting}
\end{frame}

\subsection{Reassigning Variables}
\begin{frame}[fragile]
\frametitle{Reassigning Variables}
\begin{itemize}
    \item Variables can be reassigned with new values.
    \item Example in C\#:
\end{itemize}
\begin{lstlisting}
int age = 25;
age = 30;
Console.WriteLine("New Age: " + age); // Output: New Age: 30
\end{lstlisting}
\end{frame}

\subsection{Benefits of Using Variables}
\begin{frame}
\frametitle{Benefits of Using Variables}
\begin{itemize}
    \item Variables allow you to store and manipulate data in your programs.
    \item Variables make your code more flexible and reusable.
    \item Variables can be used to store the results of calculations and function calls.
\end{itemize}
\end{frame}
