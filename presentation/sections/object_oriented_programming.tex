\section{Object-Oriented Programming}

\subsection{Classes and Objects}
\begin{frame}[fragile]
\frametitle{Classes and Objects}
\begin{itemize}
    \item Classes are blueprints for creating objects.
    \item Objects are instances of classes.
    \item Example in C\#:
\end{itemize}
\begin{verbatim}
class Person {
    public string Name;
    public int Age;

    public void Greet() {
        Console.WriteLine("Hello, my name is " + Name);
    }
}

Person person = new Person();
person.Name = "Bob";
person.Age = 25;
person.Greet(); // Output: Hello, my name is Bob
\end{verbatim}
\end{frame}

\subsection{Encapsulation}
\begin{frame}[fragile]
\frametitle{Encapsulation}
\begin{itemize}
    \item Encapsulation is the concept of hiding the internal state and requiring all interaction to be performed through an object's methods.
    \item Example in C\#:
\end{itemize}
\begin{verbatim}
class BankAccount {
    private double balance;

    public void Deposit(double amount) {
        if (amount > 0) {
            balance += amount;
        }
    }

    public double GetBalance() {
        return balance;
    }
}

BankAccount account = new BankAccount();
account.Deposit(100);
Console.WriteLine(account.GetBalance()); // Output: 100
\end{verbatim}
\end{frame}
