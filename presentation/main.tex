\documentclass{beamer}

\title{Introduction to Programming Concepts in C\#}
\author{Your Name}
\date{\today}

\begin{document}

\frame{\titlepage}

\begin{frame}
\frametitle{Table of Contents}
\tableofcontents
\end{frame}

\section{Basic Programming Concepts}
\begin{frame}
\frametitle{Basic Programming Concepts}
\begin{itemize}
    \item Variables
    \item Data Types
    \item Control Flow
\end{itemize}
\end{frame}

\begin{frame}[fragile]
\frametitle{Variables}
\begin{itemize}
    \item Variables are used to store data.
    \item Example in C\#:
\end{itemize}
\begin{verbatim}
int x = 5;
string name = "John";
\end{verbatim}
\end{frame}

\begin{frame}[fragile]
\frametitle{Data Types}
\begin{itemize}
    \item Data types define the type of data a variable can hold.
    \item Example in C\#:
\end{itemize}
\begin{verbatim}
int age = 30;
double salary = 50000.50;
bool isEmployed = true;
\end{verbatim}
\end{frame}

\begin{frame}[fragile]
\frametitle{Control Flow}
\begin{itemize}
    \item Control flow statements control the order in which instructions are executed.
    \item Example in C\#:
\end{itemize}
\begin{verbatim}
if (age > 18) {
    Console.WriteLine("Adult");
} else {
    Console.WriteLine("Minor");
}
\end{verbatim}
\end{frame}

\section{Functions and Methods}
\begin{frame}
\frametitle{Functions and Methods}
\begin{itemize}
    \item Defining and Calling Functions
    \item Passing Arguments to Parameters
\end{itemize}
\end{frame}

\begin{frame}[fragile]
\frametitle{Defining and Calling Functions}
\begin{itemize}
    \item Functions are blocks of code that perform a specific task.
    \item Example in C\#:
\end{itemize}
\begin{verbatim}
int Add(int a, int b) {
    return a + b;
}

int result = Add(3, 4);
Console.WriteLine(result); // Output: 7
\end{verbatim}
\end{frame}

\begin{frame}[fragile]
\frametitle{Passing Arguments to Parameters}
\begin{itemize}
    \item Arguments are values passed to a function when it is called.
    \item Parameters are variables that hold the values of the arguments.
    \item Example in C\#:
\end{itemize}
\begin{verbatim}
void Greet(string name) {
    Console.WriteLine("Hello, " + name);
}

Greet("Alice"); // Output: Hello, Alice
\end{verbatim}
\end{frame}

\section{Object-Oriented Programming}
\begin{frame}
\frametitle{Object-Oriented Programming}
\begin{itemize}
    \item Classes and Objects
    \item Encapsulation
\end{itemize}
\end{frame}

\begin{frame}[fragile]
\frametitle{Classes and Objects}
\begin{itemize}
    \item Classes are blueprints for creating objects.
    \item Objects are instances of classes.
    \item Example in C\#:
\end{itemize}
\begin{verbatim}
class Person {
    public string Name;
    public int Age;

    public void Greet() {
        Console.WriteLine("Hello, my name is " + Name);
    }
}

Person person = new Person();
person.Name = "Bob";
person.Age = 25;
person.Greet(); // Output: Hello, my name is Bob
\end{verbatim}
\end{frame}

\begin{frame}[fragile]
\frametitle{Encapsulation}
\begin{itemize}
    \item Encapsulation is the concept of hiding the internal state and requiring all interaction to be performed through an object's methods.
    \item Example in C\#:
\end{itemize}
\begin{verbatim}
class BankAccount {
    private double balance;

    public void Deposit(double amount) {
        if (amount > 0) {
            balance += amount;
        }
    }

    public double GetBalance() {
        return balance;
    }
}

BankAccount account = new BankAccount();
account.Deposit(100);
Console.WriteLine(account.GetBalance()); // Output: 100
\end{verbatim}
\end{frame}

\end{document}
